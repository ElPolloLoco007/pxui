
\chapter{Methodology of the Study}

\section{Technology}
One of the main goals is to establish a communication link between the back-end REST API served by statbank application and the new front-end prototype user interface. 

This means that the technology used in this project has to be a solid front-end technology that communicates well with REST API's. Other than that the technology chosen has to be accessible for a variety of users since the statbank is intended to the public. 

Code maintainment is an important factor in picking the right technology since this prototype is intended to all statistical offices.

Many technologies have been considered relevant to the project's web interface prototype.

In the end, JavaScript\footnote{\href{https://www.javascript.com/}{JavaScript}\label{javascript}} has with related advanced JavaScript frameworks has been chosen as the main programming language. JavaScript is one of the worlds most used front-end programming language\footnote{\href{http://blog.stoneriverelearning.com/top-10-programming-languages-used-in-web-development/}{Top 10 programming languages used in web development}\label{jstop10}} supported and implemented in all modern browsers with the support of new versions of the language. Addition to this, the React\footnote{\href{https://reactjs.org/}{Reactjs}\label{react}} library will be used for building the user interface. React main maintainer is Facebook.

\subsection{JavaScript}

JavaScript is one of the worlds most used front-end programming language supported and implemented in all modern browsers with the support of new versions of the language which is relevant since the user interface has to be accessible to a variety of users.

One of the advantages of using JavaScript in this project is that JavaScript is not compiled and therefore can be run immediately within the client-side browser. Not only is it an advantage for the users but makes it easy for other statistical offices to implement and try out the prototype user interface.  

The data from the PX WEB API is served in JSON\footnote{\href{https://json.org/}{JSON}\label{json}}. JavaScript is highly compatible with JSON since the syntax of JavaScript Object Notation is based on JavaScript object syntax. It  consists of a metadata part and a data part. Metadata is structured in a hierarchical node tree, where each node contains information about subnodes that are below it in the tree or, if the nodes are at the bottom of the tree structure, the data referenced by the node as well as what dimensions are available for the data at that subnode.

Other programming languages were considered, among them TypeScript\footnote{\href{https://www.typescriptlang.org/}{TypeScript}\label{typescript}} and ASP.NET\footnote{\href{https://dotnet.microsoft.com/apps/aspnet}{ASP.NET}\label{asp.net}}.

\subsection{TypeScript}

TypeScript is a strongly typed, object oriented, compiled language. TypeScript is both a language and a set of tools. TypeScript is a typed superset of JavaScript compiled to JavaScript. In other words, TypeScript is JavaScript plus some additional features. This means that TypeScript has a steeper learning curve than JavaScript with new syntax of TypeScript and strict typing.

\subsection{ASP.NET}

ASP.net is a framework for running web applications on the server. Applications that run on the server are used for processing data that you don't want to user to have access to. This project is a client-side project.


\section{Methods and techniques}
\subsection{Software engineering}
\subsection{Agile}
\subsection{Communication}
use case
\subsection{Planning}
risk analysis - object Diagram
\subsection{Modeling}
sequence diagram
\subsection{Construction}
Architecture diagram - Component diagram - tests
\subsection{Deployment}
Deployment diagram - figma

